%--------------------------------------
\chapter{Conclusion}
\label{chap_conclusion}
%--------------------------------------

MASC has been demonstrated once again as being effective at finding flaws in crypto-detectors. The work has been greatly expanded but there are still more areas for further improvement. This new extension of MASC has also reintroduced some areas of discussion initially brought to light by the original work. Since some time has passed some perspectives of these discussions have shifted.

%--------------------------------------
\section{Limitations}
\label{ch6:sec:limitations}
%--------------------------------------

MASC is still designed to help find flaws in crypto-detectors and still cannot guarantee all flaws in a crypto-detector will be found. In fact formal verification should still be done in conjunction with MASC. MASC's design from the very beginning was to allow for "systematic evaluation of crypto-detectors, which is an advancement over manually curated benchmarks." This still remains true while the extension has expanded MASC's functionality and coverage it is still not representative of all misuses. MASC is still held back by the following limitations:

1. Completeness of the Taxonomy: The same approach was used to ensure the taxonomy was comprehensive as the original work. All steps were performed carefully and utilized the same best practices found in other work. However, it is still possible that some cases or sources could be missed during extraction. The extra evaluation step was added to help ensure accuracy but it is still possible that some subtle contexts were once again missed. Up to current date though I believe that this is still, to the best of my knowledge, the most comprehensive found in recent works taxonomy in this space.

2. Focus on Generic Mutation Operators: This was a concern of the original paper since the goal was to apply as many misuses from the taxonomy as possible. This was a main area of expansion for this extension. MASC now contains many more new operators that do represent some of the more specific cases found in the taxonomy. However, the taxonomy is still not fully represented and many of the new operators were still designed to have multiple use cases. Priorities for new operators were still focused on covering more potential misuse even though the focus was exclusively on expanding the restrictive operators.

3. Focus on Java and JCA: MASC's approach is still informed by JCA and Java. No work has been done since the original paper to adapt MASC to other languages at this time.

4. Evolution of APIs: Since changes are still made to how JCA operates this may eventual lead to changes being necessary in MASC. Up to now MASC is current and the operators from the original paper do still function correctly. However, as time passes some changes may become necessary despite MASC using reflection and automated code generation to ensure flexibility. In addition, it is still an ongoing project to include more misuse cases within MASC to get closer to fully representing the current taxonomy and beyond.

5. Relative Effectiveness of Individual Operators: My research did not look into this limitation that was present in the original paper. This paper looks further into what MASC is capable of doing as a whole but does not evaluate how effective each operator is at finding flaws. This still requires its own study.

6. Consistency with the Original Work: I was in constant communication with the original authors and confirming with them how the original work was created. I incorporated feedback from them and had them review changes to the tool to help ensure as much consistency as possible. In addition, for evaluation and expansion I followed all the steps laid out by the original work to ensure that the work could stand as one whole project rather than an add-on. I believe that this was done to the best of my ability and MASC now stands as an expanded and more comprehensive tool. Since I was not directly involved in the original work there may be some small details that were not done exactly the same as the original work.

%--------------------------------------
\section{Discussion}
\label{ch6:sec:discussion}
%--------------------------------------

\textbf{Security-centric Evaluation Design:}
MASC still places a heavy focus on security centric design. From the perspective of this framework it is believed that security should come first no matter how unlikely or evasive a case may be. This is part of why some new operators were designed to create more evasive cases. This idea still clashes with the idea that designers of tools look more into a technique-centric perspective. Many of these tools are not designed with a threat model in mind or directly from a security perspective. In fact many of the tools analyze best coding practices in addition to looking for security misuses. This leads to a gap between what is expected of a tool that claims it can perform security checks.

\textbf{What is scope for Technique-centric Design:}
There is still an ongoing discussion on the space on what the scope should be. Should there be more of a focus placed on what is most common and uncommon or should it be on what can easily be computed statically vs cannot easily be computed. Even from a security perspective this is challenge since common misuses would be expected to be caught, however, new misuses pose a bigger threat due to the lack of awareness of the looming threat. Optimally most tools would like to cover all possible cases but since this is not feasible the debate is what is most important and what should be expected on crypto-detectors.

\textbf{The Need to Strengthen Crypto-Detectors:}
Many detectors make claims with the assurance that they can detect certain misuses. When tested many of these claims are proven to still fall short of expectations. If crypto-detectors claim to be able to secure your code it is important that they actually can. This means detecting uncommon cases and being held accountable for falling short. MASC has shown that it is possible to find gaps in crypto-detectors and help them improve but they are still a long way off from being able to detect hard to statically compute misuses cases. This leads to the question of should crypto-detectors be able to make the claims they do to secure your code?

\textbf{Shifting toward security centric design:}
The original work showed that there is interest in developers to make tools more secure. This was something they strived for and it was proven that some of the tools that were evaluated did improve since the first paper. This demonstrates that not only did the express interest they demonstrated interest. Even in the new tools that were evaluated when issues were reported to them they were eager to make a fix to improve their tool. Many developers have expectations that these tools help make their code secure and a lot of times are not aware of common misuses. The further these tools can expand with security centric design in mind the more developers can expect to rely on them.

\textbf{Pushing toward SARIF:}
As mentioned SARIF is a relatively new SAST output format. Some crypto-detector designers have been reached out to and expressed interest in being able to produce this format for their tool. MASC integrated this in hopes of helping to encourage more crypto-detector designers to look into this format. If there is a standard output not only does it make it easier to evaluate how well tool perform it also allows for new tools to be produced to help users evaluate their applications. Having a standard output as an expectation can lead to easier evaluation and can help create a way to fully automize the evaluation of crypto-detector. This would help users be able to easily know how reliable any tool is before they use it. In addition, if this was a standard output this would also allow tools to more easily report all misuses found. As mentioned in a prior section some crypto-detectors such as Amazon Code Guru do not provide a full report for their analysis. This is likely due to UI limitations since the whole codebase is still scanned. Having an output in the style of SARIF can help provide a full report to the user since it likely would not have to be directly consumed by the user, it would be parsed and display the output that way. 

\textbf{The Need for constant evaluation:}
As shown in the reevaluation of the original crypto-detectors some bugs or flaws have the potential to reappear in future versions of a tool. Due to the ongoing iteration of crypto-detectors it is important for them to be reevaluated. While obviously an evaluation would be necessary when a new misuse appears reevaluating the old misuses is equally important. Bugs have a tendency to reappear while refactoring and making changes and as found the same thing goes for flaws. Having a constant evaluation will help ensure that once a flaw is found and eliminated that it stays that way.

\textbf{Improvements seen due to MASC:}
As noted in the results it was found that on average the tools that were previously evaluated by MASC located more flaws than the five crypto-detectors that were evaluated for the first time. This can be attributed to a variety of factors such as age of the crypto-detector, how focused they are on security, and how they are designed. However, it is worth mentioning that the tools from the initial paper did have their flaws reported to them and clearly made some fixes. This does help demonstrate that MASC has had an impact on crypto-detectors in the industry at least to some degree. It is important that MASC continues to expand and be used to evaluate tools because it helps reveal flaws to the maintainers that they may not already be aware of and shows that users care about these ideas especially in the context of security.


%--------------------------------------
\section{Lessons Learned}
\label{ch6:sec:lessons}
%--------------------------------------

Throughout my time working on this project I went from someone who had never worked on a large software engineering project to someone conducting research and greatly expanded on the project. Throughout my time I have learned a lot about how to conduct research in a methodical way and what that means in the field of Software Engineering. I have faced many challenges throughout my work but each one has allowed me to become both a better developer and researcher. Each challenge has pushed me to obtain a new skill set. For this project I started small by simply trying to understand the codebase I was working on. As time went on and I got a further grasp I was able to build new tools such as the automated analysis. As time pushed further I was tasked with conducted research and producing my own ideas for ways to improve MASC this led to the addition of new operators and sensitivity evaluator. This project has truly helped me grow as an individual.

%--------------------------------------
\section{Conclusion}
\label{ch6:sec:conclusion}
%--------------------------------------

The creation of a tool like MASC has helped shift the mindset of the crypto-detectors designers into looking further into a more security-focused design. They have made claims and as found in this extension they have improved in some areas. There is still a long way to go and by reaching out to the designers and reporting flaws I have seen that there is interest in ensuring security. With the MASC framework existing and being known to the crypto-detector makers it is possible for them to easily perform self evaluation and continue to improve. As MASC continues to expand in the future it is possible to help push crypto-detectors into become more secure as well. With MASC existing in the same ecosystem as crypto-detectors it is possible to help push towards a more security centric design for tools and one day users can truly expect when their code is analyzed that it is secure.