%--------------------------------------
\chapter{Introduction}
\label{chap_intro}
%--------------------------------------


Cryptography is esential in making sure data is confidential and secure in the modern world. Making sure cryptography primiatives are used correclty, however, has always been difficult problem to solve wether it be in critical systems like banking or the wide spread misuse of cryptographic API found in mobile and web apps. These misuses can lead to vulnerabilities that allow for data leaks and cause sensitive data to be leaked. To solve this problem researchers have designed tools that can be integrated into the development pipeline that catch these misuses and allow developers to catch the problems before they are released to the public. These tools are known as crypto-API misuse detectors also known as crypto-detectors. These crypto detectors are vital to helping keep development of applications secure. However, these crypto-dectors are not without their own faults.
Crypto-dectors see widespread use across IDEs, corporate testing suites, by source control platforms (such as GitHub), and comercial use. Crypto-dectors are relied upon by many developers that deal with sensitive information, there is an expectation and an assumption that they are reliable. Developers utilize these crypto-dectors to help have peace of mind that their code is now more secure. However, a lot is still unknown about how effective they are at crypto-API misuse despite how eager developers are to adopt them. The MASC framework presented in the orignal paper tried to provide a solution to this problem by providing a framework to evaluate crypto-dectors beyond simple manually created benchmarks.
As described in the original paper [insert citation] MASC is "the first systematic, data-driven framework that leverage the well-founded appraoch of Mutation Analysis for evaluating Static Crypto-API misuse detectors." MASC is used by a user in a imilar manner to typical mutation analysis. MASC is capable of "mutating Android/Java apps by seeding them mutants" containing crypto-API misuse. The mutated apps can then be anaylzed by a crypto-dector and seeing which mutants are not located during the analysis reveals both design and implementation flaws in that crypto-dector.
For my thesis we took the already existing MASC framework and extended it. This was done by adding a variety of new functionality and greatly expanding the functionality that already existed. We expanded MASC in both depth and width. The updated MASC framework was also used to reevaluate  Crypto-Dectors that were evaluated by the original paper and evaluate 5 additional crypto-dectors. With the new additions to MASC and new evaluation we were able to determine: do misuses that were reported and fixed have a tendency to reappear in future builds, have crypto-dectors improved since the original paper, can MASC discover new flaws in crypto-dectors, what are the chracterisitcs of these flaws, and what is the impact of the flaws on the effectiveness of crytpo-dectors in practice?



%--------------------------------------
\section{Overview}
\label{ch1:sec:overview}
%--------------------------------------


The original MASC framework was built based on a Crypto-API Misuse Taxonomy. At the time is was the first comprehensive taxnonoy of crypto-API misuse cases containing 105 cases. These were found using a data driven process that systematially identified, studiad, and extacted misuse cases from both academic and industrial sources published from 1998-2018. This taxonomy provided the main building block for designing mutants that can emulate realistic misuses. For my thesis we expanded this taxonomy to cover papers from 2019-2022. The extended taxonomy includes ... more papers and ... more misuses.
MASC also had several Crypto-Mutation Operators and Scopes. These different abstractions were designed to allow flexible in MASC to create a large variety of feasible mutants. The threat model MASC is based on consists of three types of conditions that crypto-dectors are likely to face. In addition, MASC also consisted of usage base mutation operatos which were general operatos that contain common characteristics that can leverage a diverse set of crypto APIs to create misuse cases that can be found within the taxonomy. These operators consist of two main categories: restrictive, where a developer can only instantiate certain objects by providing values from a predefined set such Cipher.getInstance(<parameter>) and flexible which consists of operators with a large amount of extensibility such as developers can customize the hostname verification component of the SSL/TLS handshake by creating a class extending the HostnameVerifier, and overiding its verify method, with any content. The original set of operators consisted of 12 operators (6 restrictive and 6 flexible). For my thesis we focused on expanding the amount of restrictive operators since more are necessary to fully represent the taxonomy. I added ... number of restrictive operators to the total count based both on misuses that were not represented from the original taxonomy and to represent the extended taxonomy. I also added restrictive operators to further extend our T3 threat model of an evasive developer since the original work was intentially conservative with what was considered evasive. After more research it was determined that new operators could be created to more evasive and still fair to crypto-dectors. MASC also includes 3 types of mutation scopes that allow for seedinig of mutants variable fiedelty to realistic API-use and threats.
I also expanded MASC to contain several new features. These consist of adding an automated analysis tool and the sensitivity runner <change name>. The automated analysis provides MASC with a way to automatically evaluate a crypto-dectors output and determine which misuses were caught or not. It leverages SARIF files (an file type similar to JSON) to determine where misuses were. It is also designed to catch flaws with crypto-dectors based on output. It can test a variety of cases against a crypto-dector that range in complexity and will report if a simple case fails to be caught and stop running, since a more complex case cannot be feasibly caught if the simple version fails. Since all evaluation from the original paper was done by hand and was very time consuming to ensure accuracy. This tool was designed to help researchers and users save time while evaluating crypto-dectors and ensure accuracy. 
The other main new feature introduced as a part of MASC is the sensitity runner. This tool defined the different types of sensitivities  common to crypto-dectors: flow, object, context, path, and alias sensitity. Certain tools claim to be built with these sensitivities in mind. These were used to categorize all the operators contained under one or more of these categories. MASC can then be run and will generate mutants only for the specified sensitity type. This allows for a lower barrier to entry for users and to allow users to test mutants specifically against the claims of a crypto-dector. 
For the evaluation of crypto-dectors I used the same methodology as the original paper but extended the number of operators and the number of applications. I evaluated 8 of the major crypto-dectors used in the original paper and also evaluated 5 addition crypto-dectors. The original paper also evaluated the crypto-dectors with a 15 different Android/Java applications and a group of minimal cases. All of these same applications and minimal cases were used again in evaluation with the addition of 16 new Android/Java applications that were mutated using the newly extended MASC. Additionally, our findings were disclosed to the designers/maintainers of the tools. 


%--------------------------------------
\subsection{Motivation and Background}
\label{ch1:subsec:motivation}
%--------------------------------------

The motivation for extending MASC still comes from the same idea as the original motivation behind MASC, "Insecure use of cryptographic APIs is the second more common cause of software vulnerabilities after data leaks. Many developers are likely to use crypto-dectors in their development pipeline since they are not experts in the area but wish to catch vulnerabilities before releasing software. This means that the reliability of crypto-dectors and their ability to locate misuses directly impacts the security of the end user and their data. Evaluating crypto-dectors is an ongoing problem. New misuses are discovered frequently and it is important to have a tool that can keep up with new misuses. The idea behind expanding the MASC framework came from the need to keep up with new misuses and still be a reliable tool for evaluting crypto-dectors. 
The crypto-dectors evaluated in the original paper needed to be reevaluated because bugs that get fixed have a tendency to reappear in future patches. This is a common occurence in the field of software engineering. The newest versions of these tools needed to be reevaluated to see if they have made any improvements with catching misuses and ensure that flaws that were found have truly been fixed. The set of evaluated crypto-dectors was also expanded so that flaws can be reported to other widely used tools to help improve them as well. In addition since the original paper was published new tools have appeared and gained traction such as Amazon CodeGuru and SonarQube. I wanted to ensure MASC conducted an evaluation on as many crypto-dectors as possible. Since MASC's viability was proven in the original work it was important to extend and improve the work as well as extend its reach. The MASC framework, like many software engineering projects, is an ever evolving framework that is designed to be a reliable way to help evaluate the effectiveness of crypto-dectors.
