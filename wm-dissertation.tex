%  W&M PhD Dissertation LaTeX File
\documentclass[cpp,11pt]{wmthesis}
\usepackage[linesnumbered, ruled, vlined]{algorithm2e}
\usepackage{graphicx}
\usepackage{balance}
\usepackage{caption}
\usepackage{moresize}
\usepackage{changepage}
\usepackage{pbox}
\usepackage{mathtools}
\usepackage[TABBOTCAP]{subfigure}
\setlength{\textfloatsep}{2pt}
\usepackage{paralist}
\usepackage{hyperref}
\usepackage[T1]{fontenc}
\usepackage{balance}
\usepackage[dvipsnames,table,xcdraw]{xcolor}
\usepackage{multirow}
\usepackage{multicol}
\usepackage{amsmath}
\usepackage{listings}
\usepackage{setspace}
\usepackage{verbatim}
\usepackage[all]{nowidow}
\usepackage{float}
\usepackage{xspace}
\usepackage{amssymb}
\usepackage{ifthen}
\usepackage{pifont}
\usepackage{textcomp}
\usepackage{pict2e,picture}
\usepackage{url}
%\usepackage{pstricks}
\usepackage{epsfig}
\usepackage[subfigure]{tocloft}
\usepackage{amsfonts}
\usepackage{amssymb}
\usepackage{latexsym}
\usepackage{booktabs}
\usepackage{tabularx}
\usepackage{rotating}
\usepackage[numbers]{natbib}
\usepackage{mdwlist}
\usepackage{colortbl}
\usepackage[normalem]{ulem}
\renewcommand{\rmdefault}{cmr} % Arial
\renewcommand{\sfdefault}{cmr} % Arial
\renewcommand{\cftchapfont}{\rm } %no bold in toc
\renewcommand{\cftchappagefont}{\rm } %no bold in toc
\newcommand\flawtag[2]{#1\def\@currentlabel{#1}\label{#2}}
\usepackage{tikz}
\newcommand*\circled[1]{\tikz[baseline=(char.base)]{
            \node[shape=circle,draw,inner sep=0.5pt] (char) {#1};}}

\makeatletter

% this  is a easy way to add and highlight new text  ...
% just comment in/out the \tnew macro ..

\newcommand{\tnew}[1]{{\bf { #1 }} }
%\newcommand{\tnew}[1]{{ { #1 }} }

% math and theorem definition

\newcommand{\ndef}{\stackrel{\rm def}{=}}

% this is used for draft only

%\renewcommand{\baselinestretch}{2}

% just to number pages in the draft

\useunder{\uline}{\ul}{}

% nothing i.e., no-numbering final and camera ready

%\pagestyle{empty}

\newcommand{\CrashScopesp}{{\textsc CrashScope~}}
\newcommand{\CrashScopes}{{\textsc \small CrashScope's~}}
\newcommand{\CrashDroid}{{\textsc CrashDroid~}}
\newcommand{\CrashScopebf}{{ \textbf{\textsc CrashScope~}}}

\newcommand{\ReDraw}{{\textsc{\small ReDraw}}\xspace}
\newcommand{\ReDraws}{{\textsc{\small ReDraw's}}\xspace}
\newcommand{\Remaui}{{\textsc{\small Remaui}}\xspace}
\newcommand{\Remauis}{{\textsc{\small Remaui's}}\xspace}
\newcommand{\CrashScope}{{\textsc{\small CrashScope}}\xspace}
\newcommand{\MonkeyLab}{{\textsc{\small MonkeyLab}}\xspace}
\newcommand{\pixcode}{{pix2code}\xspace}

\newcommand{\GVTsp}{{\textsc{\small Gvt~}}}
\newcommand{\GVT}{{\textsc{\small Gvt}}}
\newcommand{\GVTs}{{\textsc{\small Gvt's~}}}
\newcommand{\dv}{{\textit{DV}}\xspace}
\newcommand{\dvs}{{\textit{DVs}}\xspace}
\newcommand{\gc}{{\textit{GC}}\xspace}
\newcommand{\gcs}{{\textit{GCs}}\xspace}
\newcommand{\mgc}{{\textit{M-GC}}\xspace}
\newcommand{\mgcs}{{\textit{M-GCs}}\xspace}
\newcommand{\igc}{{\textit{I-GC}}\xspace}
\newcommand{\igcs}{{\textit{I-GCs}}\xspace}


\newcommand{\ReDrawAbs}{{R}{\ssmall E}{D}{\ssmall RAW}\xspace}




\newsavebox\CBox
\newlength\CLength
\def\circled#1{\sbox\CBox{#1}%
  \ifdim\wd\CBox>\ht\CBox \CLength=\wd\CBox\else\CLength=\ht\CBox\fi
    \makebox[1.2\CLength]{\makebox(0,0.9\CLength){\put(0,0){\circle{1.3\CLength}}}%
    \makebox(0,1.0\CLength){\put(-.5\wd\CBox,0){#1}}}}

\def\circledlong#1{\sbox\CBox{#1}%
  \ifdim\wd\CBox>\ht\CBox \CLength=\wd\CBox\else\CLength=\ht\CBox\fi
    \makebox[1.2\CLength]{\makebox(0,0.6\CLength){\put(0,0){\circle{1.3\CLength}}}%
    \makebox(0,0.6\CLength){\put(-.5\wd\CBox,0){#1}}}}

\lstset{
	basicstyle=\footnotesize\ttfamily,
	breaklines=true,
    frame=tb, % draw a frame at the top and bottom of the code block
    tabsize=4, % tab space width
    showstringspaces=false, % don't mark spaces in strings
    numbers=left, % display line numbers on the left
    commentstyle=\color{Red}, % comment color
    keywordstyle=\color{blue}, % keyword color
    stringstyle=\color{OliveGreen}, % string color
	xleftmargin=.25in %align numbers to left side
}

\newboolean{showcomments}


\setboolean{showcomments}{true}

\ifthenelse{\boolean{showcomments}}
  {\newcommand{\nb}[2]{
    \fbox{\bfseries\sffamily\scriptsize#1}
    {\sf\small$\blacktriangleright$\textit{#2}$\blacktriangleleft$}
   }
   \newcommand{\cvsversion}{\emph{\scriptsize$-$Id: macro.tex,v 1.9 2005/12/09 22:38:33 giulio Exp $}}
  }
  {\newcommand{\nb}[2]{}
   \newcommand{\cvsversion}{}
  }


\newcommand{\ie}{\textit{i.e.},\xspace}
\newcommand{\eg}{\textit{e.g.},\xspace}
\newcommand{\etc}{\textit{etc.}\xspace}
\newcommand{\etal}{\textit{et al.}\xspace}
\newcommand{\aka}{\textit{a.k.a.}\xspace}

\newcommand\NEW[1]{\nb{NEW}{#1}}

\newcommand\operator[2]{{\bf OP$_{#1}$: {\em {#2}} -- }}
\newcommand\opnumber[1]{{{\bf OP}$_{#1}$}}

\newcommand{\projtodo}[1]{{\color{red}{\bf TODO:}#1}}

\newcommand*{\img}[1]{%
    \raisebox{-.2\baselineskip}{%
        \includegraphics[
        height=0.85\baselineskip,
        width=0.85\baselineskip,
        keepaspectratio,
        ]{#1}%
    }%
}



\setboolean{showcomments}{true}



\newcommand{\addnewline}{\\}

\newcommand{\cancel}[1]{{\leavevmode\color{RubineRed}{\sout{\xspace#1}}}}
\newcommand{\edit}[2]{{\leavevmode\color{RubineRed}{\sout{#1}}}{\color{blue}{\xspace#2}}}
\newcommand{\editl}[2]{{\leavevmode\color{RubineRed}{\addnewline\sout{#1}}}{\color{blue}{\addnewline\xspace#2\addnewline}}}
\newcommand{\rewrite}[2]{{\leavevmode\color{RubineRed}{\sout{#1}}}{\color{Green}{\arrow\xspace#2}}}


\newcommand{\add}[1]{{\leavevmode\color{blue}{#1}}}
\newcommand{\addamit}[1]{{\leavevmode\color{black}{#1}}}
\newcommand{\addnew}[1]{{\leavevmode\color{black}{#1}}}

\newcommand{\addl}[1]{{\leavevmode\color{Green}{\addnewline+ \xspace#1\addnewline}}}
\newcommand{\remove}[1]{{\leavevmode\color{red}{\xspace#1}}}
\newcommand{\removel}[1]{{\leavevmode\color{red}{\addnewline- \xspace#1\addnewline}}}


\newcommand{\grayme}[1]{{\leavevmode\color{gray}{\xspace#1}}}


\newcommand{\recycler}{\textsf{RecyclerView}\xspace}


\newcommand\finding[1]{\vspace{0.25em}\noindent\textsf{\bf Finding {#1}.}}
\newcommand\fnumber[1]{{\bf F{#1}}}
\newcommand\opnumbernormal[1]{{{OP}$_{#1}$}}



\newcommand{\arrow}{{$\rightarrow$}\xspace}
\newcommand\inline[1]{{\lstinline[keywordstyle=\color{black},basicstyle=\scriptsize\ttfamily,stringstyle=\color{black}]{#1}}}
\newcommand\inlinesmall[1]{{\lstinline[keywordstyle=\color{black},basicstyle=\small\ttfamily,stringstyle=\color{black}]{#1}}}



\newcommand{\boxme}[1]{{
\begin{tcolorbox}[enhanced,skin=enhancedmiddle,borderline={1mm}{0mm}{MidnightBlue}]
    \textbf{Insight: } #1 \end{tcolorbox}
}}



\newcommand\fix[1]{{\color{blue} \nb{FIX THIS}{#1}}}
\newcommand\blue[1]{{\color{blue}{#1}}}
\newcommand{\here}{{\color{blue} \nb{***}{CONTINUE HERE}}}
\newcommand{\REF}{{\color{red} \textbf{[REFS]}}\xspace}
\newcommand{\xy}{{\color{red} \textbf{XY}}\xspace}
\newcommand\tops[1]{{\color{blue}{#1}}}
\newcommand\alert[1]{{\color{red}{#1}}}


\newcommand{\target}{\textit{target tool}\xspace}
\newcommand{\targets}{\textit{target tools}\xspace}
\newcommand{\behavior}{\textit{target behavior}\xspace}


\newcommand{\mplus}{{\sc MDroid+}\xspace}

\newcommand{\secref}[1]{\S\ref{#1}\xspace}
\newcommand{\figref}[1]{Fig.~\ref{#1}\xspace}
\newcommand{\tabref}[1]{Table~\ref{#1}\xspace}
\newcommand{\Phase}{{\sc Phase}\xspace}
\newcommand{\Phases}{{\sc Phase's~}\xspace}

\newcommand{\emphquote}[1]{{\emph{`#1'}}\xspace}
\newcommand{\emphdblquote}[1]{{\emph{``#1''}}\xspace}

\newcommand{\emphbrack}[1]{\emph{[#1]}\xspace}

\newcommand{\subj}{\emphbrack{subject}}
\newcommand{\act}{\emphbrack{action}}
\newcommand{\obj}{\emphbrack{object}}
\newcommand{\prep}{\emphbrack{preposition}}
\newcommand{\objtwo}{\emphbrack{object2}}

\newcommand*\ciclednum[1]{\raisebox{.5pt}{\textcircled{\raisebox{-.9pt}
{#1}}}}

\newcommand\codel[1]{\begin{verbatim}{#1}\end{verbatim}}










\newcommand{\tool}{{\tt MASC}\xspace}
\newcommand{\tools}{{\tt MASC's}\xspace}
\newcommand{\mdroid}{{\small MDroid$+$}\xspace}
\newcommand{\crashscope}{{\small CrashScope}\xspace}
\newcommand{\detector}{crypto-detector\xspace}
\newcommand{\detectors}{crypto-detectors\xspace}

\newcommand{\cipher}{\texttt{Cipher}\xspace}
\newcommand{\ciphergetinstance}{\texttt{\small Cipher.getInstance(<parameter>)}\xspace}
\newcommand{\messageDigestInstance}{\texttt{\small MessageDigest.getInstance(<parameter>)}\xspace}
\newcommand{\messageDigest}{\texttt{\small MessageDigest}\xspace}
\newcommand{\des}{\texttt{des}\xspace}
\newcommand{\DES}{\texttt{DES}\xspace}
\newcommand{\AES}{\texttt{AES}\xspace}
\newcommand{\ECB}{\texttt{ECB}\xspace}
\newcommand{\MDFIVE}{\texttt{MD5}\xspace}
\newcommand{\muse}{$\mu$SE\xspace}


\newcommand{\trustManager}{\texttt{\small TrustManager}\xspace}
\newcommand{\xtrustManager}{\texttt{\small X509TrustManager}\xspace}
\newcommand{\extendedxtrustManager}{\texttt{\small X509ExtendedTrustManager}\xspace}
\newcommand{\getAcceptedIssuers}{\texttt{\small getAcceptedIssuers}\xspace}
\newcommand{\checkServerTrusted}{\texttt{\small checkServerTrusted}\xspace}
\newcommand{\checkClientTrusted}{\texttt{\small checkClientTrusted}\xspace}

\newcommand{\certificateException}{\texttt{\small CertificateException}\xspace}

\newcommand{\hostnameVerifier}{\texttt{\small HostnameVerifier}\xspace}
\newcommand{\trycatch}{\texttt{\small try-catch}\xspace}

\newcommand{\ivparameterspec}{\texttt{IvParameterSpec}\xspace}

\newcommand{\cryptoguard}{CryptoGuard\xspace}
\newcommand{\cryptoguardmultidex}{{$2,709$}\xspace}
\newcommand{\cryptoguarddownload}{{$4,353$}\xspace}
\newcommand{\cryptoguardtotal}{{$6,181$}\xspace}
\newcommand{\cryptoguardpercent}{{$62.23$\%}\xspace}
\newcommand{\cryptoguardandroiddottotal}{{$673$}\xspace}
\newcommand{\cryptoguardandroiddot}{{$383$}\xspace}
\newcommand{\cryptoguardendswithandroid}{{$290$}\xspace}
\newcommand{\cryptoguardandroiddottotalpercent}{{$10.89$\%}\xspace}
\newcommand{\androiddot}{{\texttt{android.}}\xspace}


\newcommand{\totalmisuses}{{$105$}\xspace}
\newcommand{\totaloperators}{{$12$}\xspace}

\newcommand{\crysl}{CrySL\xspace}
\newcommand{\cognicrypt}{CogniCrypt\xspace}
\newcommand{\xanitizer}{Xanitizer\xspace}
\newcommand{\coverity}{Tool$_X$\xspace}
\newcommand{\spotbug}{SpotBugs\xspace}
\newcommand{\spotbugfull}{SpotBugs with FindSecBugs\xspace}
\newcommand{\qark}{QARK\xspace}
\newcommand{\qpid}{Apache Qpid Broker-J\xspace}
\newcommand{\shiftleft}{ShiftLeft\xspace}
\newcommand{\codeqlgcs}{Github Code Security\xspace}
\newcommand{\codeqllgtm}{LGTM\xspace}

\newcommand{\codeguru}{Amazon Code Guru\xspace}
\newcommand{\sonarqube}{SonarQube\xspace}
\newcommand{\snyk}{Snyk\xspace}
\newcommand{\codiga}{Codiga\xspace}
\newcommand{\deepsource}{DeepSource\xspace}
\newcommand{\newcodeql}{Machine Learning CodeQl\xspace}
\newcommand{\cryptoguardupdate}{CryptoGuard version \xspace}
\newcommand{\cognicryptupdate}{CogniCrypt version \xspace}
\newcommand{\spotbugsupdate}{SpotBugs with FindSecBugs version \xspace}
\newcommand{\coverityupdate}{Coverity version \xspace}
\newcommand{\qarkupdate}{QARK version \xspace}
\newcommand{\shiftleftupdate}{ShiftLeft version \xspace}
\newcommand{\codeqlversion}{GitHub Code Security version \xspace}
\newcommand{\newlgtm}{LGTM version \xspace}

\newcommand{\coverityshort}{TX\xspace}
\newcommand{\xanitizershort}{XT\xspace}
\newcommand{\cryptoguardshort}{CG\xspace}
\newcommand{\shiftleftshort}{SL\xspace}
\newcommand{\qashort}{QA\xspace}
\newcommand{\codeqlgcsshort}{GCS\xspace}
\newcommand{\codeqllgtmshort}{LGTM\xspace}
\newcommand{\cryslshort}{CL\xspace}
\newcommand{\sportbugsshort}{SB\xspace}
\newcommand{\cognicryptshort}{CC\xspace}

\newcommand{\codegurushort}{AC\xspace}
\newcommand{\sonarqubeshort}{SQ\xspace}
\newcommand{\snykshort}{SY\xspace}
\newcommand{\codigashort}{CD\xspace}
\newcommand{\deepsourceshort}{DS\xspace}
\newcommand{\newcodeqlshort}{NGSC\xspace}
\newcommand{\newcryptoguardshort}{NCG\xspace}
\newcommand{\newcognicryptshort}{NCC\xspace}
\newcommand{\newspotbugsshort}{NSB\xspace}
\newcommand{\newcoverityshort}{NTX\xspace}
\newcommand{\newqarkshort}{NQA\xspace}
\newcommand{\newshiftleftshort}{NSL\xspace}
\newcommand{\newcqversion}{NLGTM\xspace}
\newcommand{\newlgtmshort}{NGCS\xspace}



\newcommand{\countflaws}{{$19$}\xspace}
\newcommand{\countFlawClasses}{{$5$}\xspace}
\newcommand{\countFlawClassesText}{{five}\xspace}


\newcommand{\no}{\remove{\ding{55}}}
\newcommand{\ye}{\ding{51}}
\newcommand{\pr}{\texttt{pr}}
\newcommand{\na}{\texttt{-}}
\newcommand{\np}{\text{\O}}



\newcommand{\fcincomplete}{Flaw Class 1 (FC1): Incomplete Analysis of Target Code\xspace}
\newcommand{\fcdifferentcase}{Flaw Class 1 (FC1): String Case Mishandling\xspace}
\newcommand{\fcvalueresoluion}{Flaw Class 2 (FC2): Incorrect Value Resolution\xspace}
\newcommand{\fcomplexinheritance}{Flaw Class 3 (FC3): Incorrect Resolution of Complex Inheritance and Anonymous Objects\xspace}
\newcommand{\fcgenericnoise}{Flaw Class 4 (FC4): Insufficient Analysis of Generic Conditions in Extensible Crypto-APIs\xspace}%
\newcommand{\fcspecificnoise}{Flaw Class 5 (FC5): Insufficient Analysis of Context-specific, Conditions in Extensible Crypto-APIs\xspace}

\newcommand{\totalMutantApplications}{{$27$}\xspace}
\newcommand{\totalMutations}{{$20,303$}\xspace}
\newcommand{\totalMutationsAndroid}{{$2,515$}\xspace}
\newcommand{\totalMutationsJava}{{$17,788$}\xspace}
\newcommand{\totalReachMutant}{{$20,165$}\xspace}
\newcommand{\testedApps}{{$17$}\xspace}
\newcommand{\testedAndroidApps}{{$13$}\xspace}
\newcommand{\testedJavaComponents}{{$4$}\xspace}
\newcommand{\testedCryptoApps}{{$7$}\xspace}
\newcommand{\totalToolsUsed}{{$9$}\xspace}
\newcommand{\totalToolsUsedText}{{nine}\xspace}


\newcommand{\mainScope}{main\xspace}
\newcommand{\similarityScope}{similarity scope\xspace}
\newcommand{\exhaustiveScope}{exhaustive scope\xspace}
\newcommand{\totalMisuseCases}{{$105$}\xspace}
\newcommand{\newMisuseCases}{{$4$}\xspace}
\newcommand{\implementedMisuseCases}{{$19$}\xspace}


\newcommand{\dexlib}{{\texttt{dexlib2}}\xspace}



\newcommand{\TsmallCaseParameter}                   {\ye & \no & \np & \ye & nr & \ye & \ye & \ye & nr & \np & \ye & \ye \\}
\newcommand{\TvalueInVariable}                      { \ye & \ye & \np & \pr & nr & \ye & \ye & \ye & nr & \np & \ye & \pr \\}
\newcommand{\TsecureParameterReplaceInsecure}       {\no & \no & \np & \no & nr & \pr & \no & \no & nr & \np & \no & \ye \\}
\newcommand{\TinsecureParameterReplaceInsecure}     {\no & \no & \np & \no & nr & \ye & \no & \no & nr & \np & \no & \no \\}
\newcommand{\TstringCaseTransform}                  {\no & \no & \np & \no & nr & \ye & \no & \ye & nr & \np & \ye & \ye \\}
\newcommand{\TnoiseReplace}                         {\no & \no & \np & \no & nr & \pr & \no & \no & nr & \np & \no & \no \\}
\newcommand{\TparameterFromMethodChaining}          { \no & \no & \np & \no & nr & \no & \no & \no & nr & \np & \no & \ye \\}
\newcommand{\TdeterministicByteFromCharacterConcat} {\ye & \no & \np & \np & nr & \no & \ye & \ye & nr & \np & \ye & \np \\}
\newcommand{\TpredictableByteFromSystemAPI}         {\ye & \no & \np & \np & nr & \no & \ye & \ye & nr & \np & \ye & \np \\}

\newcommand{\TXExtendedTrustManager}                {\ye & \ye & \np & \ye & nr & \no & \na & \ye & nr & \pr & \ye & \np \\}
\newcommand{\TXTrustManagerSubType}                 {\no & \ye & \np & \no & nr & \no & \na & \ye & nr & \pr & \ye & \np \\}
\newcommand{\TIntHostnameVerifier}                  {\no & \ye & \np & nr & nr & \no & \na & \ye & nr & \na & \ye & \np\\}
\newcommand{\TAbcHostnameVerifier}                  {\no & \no & \np & \pr & nr & \no & \na & \ye & nr & \na & \ye & \np \\}

\newcommand{\TXTrustManagerGenericConditions}       {\ye & \ye & \np & \no & nr & \no & \pr & \ye & nr & \no & \ye & \np \\}
\newcommand{\TIntHostnameVerifierGenericCondition}  {nr & nr & nr & nr & nr & \no & \na & \ye & nr & \na & \ye & \np \\}
\newcommand{\TAbcHostnameVerifierGenericCondition}  {\no & \no & \np & \no & nr & \no & \na & \ye & nr & \na & \ye & \np \\}


\newcommand{\TXTrustManagerSpecificConditions}      {\no & \ye & \np & \no & nr & \no & \pr & \ye & nr & \no & \ye & \np \\}
\newcommand{\TIntHostnameVerifierSpecificCondition} {\no & \no & \np & nr & nr & \no & \na & \ye & nr & \na & \ye & \np \\}
\newcommand{\TAbcHostnameVerifierSpecificCondition} {\no & \no & \np & \no & nr & \no & \na & \ye & nr & \na & \ye & \np \\}




% The wmthesis class is based on the latex report class whic
% only indents paragraphs if they immediately follow other paragraphs.  The
% dissertation lady says this is wrong.  I tend to give more credence
% to Dr. Knuth (author of TeX) on this issue, since the other way looks really
% crappy.  If you want the first line of every paragraph indented,
% uncomment the next line to include the indentfirst package. -- rem
% \usepackage{indentfirst}
% Not sure if this is still an option -- Ruth

\def\BEGINITEMIZE{\begin{itemize}}
\def\ENDITEMIZE{\end{itemize}}

\def\defaultpenalty{1000} \clubpenalty=\defaultpenalty
\widowpenalty=\defaultpenalty

%%%%%%%%%%%%%%%%%%%%%%%%%%%%%%%%%%%
%%   I put all of my specially defined commands in this file to ensure
%%   that I maintain consistency since I changed some notation 
%%   usage between publications. Feel free to delete and/or modify
%%   to suit your purpose.
%%%%%%%%%%%%%%%%%%%%%%%%%%%%%%%%%%%
%%--Some general purpose macros.
%%----------------------------------------------------------------
% Usage:
%     \inputfig{filename} 
%        or
%     \inputfig[scaling_factor]{filename} 
%        filename.ps    is expected to be in the directory ../figs
%        scaling_factor the amount of scaling to be applied 
%                       (a decimal fraction between 0.0 and 1.0
%           
\newcommand{\inputfig}[2][\empty]{ %
   \begin{center} %
   	\ifx\empty#1 \includegraphics{../figs/#2}
	\else\scalebox{#1}{\includegraphics{../figs/#2}}\fi
   \end{center}}

% Usage: 
%     \inputplot{filename}
%        filename.ps is expected to be in the directory ../plots/ps
\newcommand{\inputplot}[1]{
   \begin{center}\includegraphics{../plots/ps/#1.ps}\end{center}}

%%----------------------------------------------------------------
% Set the title that will be printed on the Contents page
%%----------------------------------------------------------------
% The negative vspace is used to make sure that only one line is
% between the title and the first line for each of these pages.
\renewcommand{\contentsname}{\begin{center}\Large\normalfont TABLE OF CONTENTS\vspace{-.5in}\end{center}}
\renewcommand\listfigurename{\begin{center}\Large\normalfont LIST OF FIGURES\vspace{-.35in}\end{center}}
\renewcommand\listtablename{\begin{center}\Large\normalfont LIST OF TABLES\vspace{-.35in}\end{center}}

%%----------------------------------------------------------------
%%----------------------------------------------------------------

\begin{document}
\doublespacing

%%--Set thesis info.
%%--*IMPORTANT* Title cannot be in ALL CAPS
\thesisTitle{A Reevaluation of Why Crypto-detectors Fail: A Systematic Revaluation of Cryptographic Misuse Detection Techniques}
\thesisAuthor[Scott Marsden]{Scott Marsden}
\thesisMonth{April}
\thesisYear{2023}
\thesisAdvisor{Professor Denys Poshyvanyk}
% location and degrees added 
% note that the degree should be spelled out, not abbreviated
\thesisLocation{Sherman Oaks, California, United States of America}
\thesisDegreeOne{Bachelor of Science, Elon University, 2020}
\thesisDegreeTwo{Master of Science, The College of William \& Mary, 2023}
%\thesisDegreeThree{Master of Science, College of William and Mary, 2015}
\thesisCommittee[Computer Science]{\ThesisAdvisor}
\thesisCommittee[Computer Science]{Assistant Professor Dmitry Evtyushkin}
\thesisCommittee[Computer Science]{Assistant Professor Adwait Nadkarni}

\thesisDedication{Insert heartfelt dedication here...}

%%-- Insert contents of acknowledge.tex and abstract.tex.  Don't
%%forget to check these files for formatting hints.
\thesisAcknowledge{acknowledge}
\thesisAbstract{abstract} 

%%--Create the dissertation Prolog
\makeThesisProlog

%%--Include the dissertation chapters.
\include{Chapter-Introduction}
\include{Chapter-Framework}
\include{Chapter-Implementation}
\include{Chapter-Evaluation}
\include{Chapter-Results}
\include{Chapter-Conclusion}
%bibliographystyle{wmbib}
%bibliography{extracted.bib}

% Start calling the chapters Appendices
\appendix
\chapter{Appendix A}


%-----------------------------------
\section{Code Snippets}
\label{appendixA:codesnippets}
%-----------------------------------

\begin{lstlisting}[frame=tb,caption={{\small Method Chaining (\opnumber{5}).}},     label={lst:method_chain},language=java]
    Class T { String algo="AES/CBC/PKCS5Padding";
    T mthd1(){ algo = "AES";  return this;} T mthd2(){ algo="DES"; return this;} }
    Cipher.getInstance(new T().mthd1().mthd2());
    \end{lstlisting}
    \vspace{-0.25em}
    
    \begin{lstlisting}[frame=tb,caption={\small Predictable/Non-Random Derivation of Value (\opnumber{6})}, label={lst:bad_derivation_operator},language=java]
    val = new Date(System.currentTimeMillis()).toString();
    new IvParameterSpec(val.getBytes(),0,8);}
    \end{lstlisting}
    \vspace{-0.25em}
    
    \begin{lstlisting}[frame=tb,caption={{\small Exception in an {\em always-false} condition block (\opnumber{7}).}}, label={lst:conditional_exception},language=java]
    void checkServerTrusted(X509Certificate[] x, String s)
    throws CertificateException {
    if (!(null != s && s.equalsIgnoreCase("RSA"))) {
         throw new CertificateException("not RSA");}
    \end{lstlisting}
    \vspace{-0.25em}
    
    \begin{lstlisting}[frame=tb,caption={\small False return within an {\em always true} condition block (\opnumber{8}).}, label={lst:condition_return},language=java]
    public boolean verify(String host, SSLSession s) {
      if(true || s.getCipherSuite().length()>=0)}
        return true;} return false;}
    \end{lstlisting}
    \vspace{-0.25em}
    
    \begin{lstlisting}[frame=tb,caption={\small Implementing an Interface with no overridden methods.}, label={lst:abstract_extension_empty},language=java]
    interface ITM extends X509TrustManager { }
    abstract class ATM implements X509TrustManager { }
    \end{lstlisting}
    \vspace{-0.25em}
    
    \begin{lstlisting}[frame=tb,caption={Inner class object from Abstract type ({\bf OP$_{12}$})}, label={lst:abstract_inner_class},language=java]
    new HostnameVerifier(){
      public boolean verify(String h, SSLSession s) {
        return true; } };
    \end{lstlisting}
    \vspace{-0.25em}
    
    
    \begin{lstlisting}[frame=tb,caption={\small Anonymous Inner Class Object of \texttt{X509ExtendedTrustManager} (\ref{flaw:X509ExtendedTrustManager})}, label={lst:aic_X509ExtendedTrustManager},language=java]
    new X509ExtendedTrustManager(){
      public void checkClientTrusted(X509Certificate[] chain, String a) throws CertificateException {}
      public void checkServerTrusted(X509Certificate[] chain, String authType)throws CertificateException {}
      public X509Certificate[] getAcceptedIssuers() {return null;} ...};
        \end{lstlisting}
        \vspace{-0.25em}
    
     \begin{lstlisting}[frame=tb,caption={\small Specific Condition in \inlinesmall{checkServerTrusted} method (\ref{flaw:X509TrustManagerSpecificConditions}) }, label={lst:specific_condition_trustmanager},language=java]
    void checkServerTrusted(X509Certificate[] certs, String s)
       throws CertificateException {
     if (!(null != s || s.equalsIgnoreCase("RSA") || certs.length >= 314)) {
       throw new CertificateException("Error");}}
        \end{lstlisting}
        \vspace{-0.25em}
    
    \begin{lstlisting}[frame=tb,caption={\small Anonymous Inner Class Object of An Empty Abstract Class that implements \hostnameVerifier}, label={lst:aic_empty_ext_abstract_hostname},language=java]
    abstract class AHV implements HostnameVerifier{} new AHV(){
      public boolean verify(String h, SSLSession s) 
          return true;}};
        \end{lstlisting}
        \vspace{-0.25em}
    
        \begin{lstlisting}[frame=tb,caption={Anonymous inner class object with a vulnerable \checkServerTrusted method (F13)}, label={lst:aic_x509tm},language=java]
    abstract class AbstractTM implements X509TrustManager{} new AbstractTM(){
      public void checkServerTrusted(X509Certificate[] chain, String authType) throws CertificateException {}
        public X509Certificate[] getAcceptedIssuers() {return null;}}};
        \end{lstlisting}
        \vspace{-0.25em}
    
    \begin{lstlisting}[frame=tb,caption={\small Anonymous Inner Class Object of an Interface that extends \hostnameVerifier}, label={lst:aic_empty_ext_interface_hostname},language=java]
    interface IHV extends HostnameVerifier{} new IHV(){
      public boolean verify(String h, SSLSession s) return true;}};
        \end{lstlisting}
        \vspace{-0.25em}
        
    
        \begin{lstlisting}[frame=tb,caption={\small Misuse case requiring a trivial new operator}, label={lst:trivial},language=java]
    KeyGenerator keyGen = KeyGenerator.getInstance("AES");
    keyGen.init(128); SecretKey secretKey=keyGen.generateKey();
        \end{lstlisting}
        \vspace{-0.25em}
    
        \begin{lstlisting}[frame=tb,caption={\small \cryptoguard's code ignoring names with "android"}, label={lst:android-dot},language=java]
    if (!className.contains("android."))
        classNames.add(className.substring(1, className.length() - 1)); return classNames;
     \end{lstlisting}
        \vspace{-0.25em}
        
        \begin{lstlisting}[frame=tb,caption={\small Generic Conditions in \checkServerTrusted{}}, label={lst:xtrustManagerGenericConditions},language=java]
    if(!(true || arg0==null || arg1==null)) {
        throw new CertificateException();}
    \end{lstlisting}
    \vspace{-0.25em}
    
    
    \begin{lstlisting}[frame=tb,caption={\small Transformation String formation in Apache Druid similar to \fnumber{2} which uses \AES in \texttt{CBC} mode with \texttt{PKCS5Padding}, a configuration that is known to be a misuse~\cite{FBX+17,owasp:mislabel}.}, label={lst:transformationstring:apache:druid},language=java]
    this.name = name == null ? "AES" : name;
    this.mode = mode == null ? "CBC" : mode;
    this.pad = pad == null ? "PKCS5Padding" : pad;
    this.string = StringUtils.format(
        "%
    \end{lstlisting}
    \vspace{-0.25em}

    \begin{lstlisting}[frame=tb,caption={\small Iterative Method Chaining}, label={lst:iterativechaining},language=java]
        Class T {
        int i = 0;
        cipher = "AES/GCM/NoPadding";
        public void A(){
            cipher = "AES/GCM/NoPadding";
        }
        public void B(){
            cipher = "AES/GCM/NoPadding";
        }
        public void C(){
            cipher = "AES/GCM/NoPadding";
        }
        public void D(){
            cipher = "AES";
        }
        public String getVal(){
            return cipher
        } }


        Cipher.getInstance(new T().A().B().C().D().getVal() ) ;
        \end{lstlisting}
        \vspace{-0.25em}
    

        \begin{lstlisting}[frame=tb,caption={\small Iterative Conditionals}, label={lst:iterativeconditionals},language=java]
            Class T {
            int i = 0;
            cipher = "AES/GCM/NoPadding";
            public void A(){
            if ( i == 0){
                if (i == 0){
                    if(i == 0){
                        cipher = "AES";
                    }
                    else{
                        cipher = "AES/GCM/NoPadding";
                    }
                }
                else{
                    cipher = "AES/GCM/NoPadding";
                    }
            } else{
                cipher = "AES/GCM/NoPadding";

            }}
            
            public String getVal(){
                return cipher
            } }
    
    
            Cipher.getInstance(new T().A().getVal() ) ;
                
            \end{lstlisting}
            \vspace{-0.25em}


\begin{lstlisting}[frame=tb,caption={\small Iterative Conditionals}, label={lst:iterativeconditionals},language=java]
            Class T {
            int i = 0;
            cipher = "AES/GCM/NoPadding";
            public void A(){
            if ( i == 0){
                if (i == 0){
                    if(i == 0){
                        cipher = "AES";
                    }
                    else{
                        cipher = "AES/GCM/NoPadding";
                    }
                }
                else{
                    cipher = "AES/GCM/NoPadding";
                    }
            } else{
                cipher = "AES/GCM/NoPadding";

            }}
            
            public String getVal(){
                return cipher
            } }
    
    
            Cipher.getInstance(new T().A().getVal() ) ;
                
            \end{lstlisting}
            \vspace{-0.25em}


            \begin{lstlisting}[frame=tb,caption={\small Method Builder}, label={lst:methodbuilder},language=java]
                Class T {
                int i = 0;
                cipher = "AES/GCM/NoPadding";
                public String A(){
                    return "D";
                }
                public String B(){
                    return "E";
                }
                public String C(){
                    return "S";
                }
                public void add(){
                    cipher = A() + B() + C();
                }
                public String getVal(){
                    return cipher
                } }
        
        
                Cipher.getInstance(new T().add().getVal() ) ;
                \end{lstlisting}
                \vspace{-0.25em}
            

                \begin{lstlisting}[frame=tb,caption={\small Object Sensitive, using the object created in Listing A.1}, label={lst:ObjectSensitivity},language=java]
                    T secure = new T();
                    T insecure = new T().mthd2();
                    secure = insecure;
                    Cipher.getInstance(secure.getVal());
                    
                        
                    \end{lstlisting}
                    \vspace{-0.25em}

\begin{lstlisting}[frame=tb,caption={\small Build Variable}, label={lst:buildvariable},language=java]
    String cryptoVariable = "AES";
    char[] cryptoVariable1  = cryptoVariable.toCharArray();
    javax.crypto.Cipher.getInstance(String.valueOf(cryptoVariable1));
                        
                            
\end{lstlisting}
\vspace{-0.25em}

\begin{lstlisting}[frame=tb,caption={\small Substring}, label={lst:substring},language=java]
    javax.crypto.Cipher.getInstance("secureParamAES".substring(11));
                        
                            
\end{lstlisting}
\vspace{-0.25em}

\begin{lstlisting}[frame=tb,caption={\small Static Keystore}, label={lst:statickeystore},language=java]
    byte[] cryptoTemp = "12345678".getBytes();
    javax.crypto.spec.IvParameterSpec ivSpec = new javax.crypto.spec.IvParameterSpec.getInstance(cryptoTemp,"AES");
                        
                            
\end{lstlisting}
\vspace{-0.25em}
        



%\input{appendixB}
%\input{appendixC}


%--List of references not actually cited in the document.
\nocite{*}

%--Include the bibliography
\makeThesisBib{references}
%\bibliographystyle{IEEEtran}
%\bibliography{references}
%% I did not include the following in my dissertation, so I have no knowledge
%% of their formatting compliance. -- Ruth
%\makeThesisVita{vita}
%%\makeUMIAbstract{abstract}
\end{document}
