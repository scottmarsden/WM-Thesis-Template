\begin{large}
\noindent 
\begin{flushleft}
The correct use of cryptography is central to ensuring data security in modern software systems. Hence, several academic and commercial static analysis tools have been developed for detecting and mitigating crypto-API misuse. While developers are optimistically adopting these crypto-API misuse detectors (or crypto-detectors) in their software development cycles, this momentum must be accompanied by a rigorous understanding of their effectiveness at finding crypto-API misuse in practice. The original paper presents the MASC framework, which enables a systematic and data-driven evaluation of crypto-detectors using mutation testing. MASC was grounded in a comprehensive view of the problem space by developing a data-driven taxonomy of existing crypto-API misuse, containing 105 misuse cases organized among nine semantic clusters. 12 generalizable usagebased mutation operators were developed and three mutation scopes that can expressively instantiate thousands of compilable variants of the misuse cases for thoroughly evaluating crypto-detectors. Using MASC, nine major crypto-detectors were evaluated and 19 unique, undocumented flaws that severely impact the ability of crypto-detectors to discover misuses in practice were found.
\end{flushleft}

\begin{flushleft}
For my thesis I built upon this previous research and greatly expanded the MASC framework. MASC was expanded in all areas by adding new functionality, new operators, new misuses, and expanding the taxonomy. In addition, Ireevaluated the most up to date versions of the original 9 crypto-detectors and evaluated 5 additional crypto-dectors. On top of this I also doubled the amount of applications I used to evaluate the tools.
\end{flushleft}

\begin{flushleft}
The new functionality that was added to MASC includes a tool for users to obtain mutants based on a type of static analysis sensitivities such as: path sensitivity, alias sensitivity, context sensitivity, object sensitivity, and flow sensitivity. I researched and defined all these sensitivities and categorized each of the existing and new operators to fit under these sensitivities. This was designed to help expand user usability of the tool. In addition Ialso added a tool to parse SARIF output (a type of output produced by crypto-detectors) and determine which mutants were caught. This tool was designed to help with analyzing crypto-detectors and speed up the analysis time. In the original paper all results were determined by hand.
\end{flushleft}
\begin{flushleft}
I expanded functionality of existing operators so that they could be more flexible and have new features. I also added 7 brand new operators to the group of existing operators. These operators allow for MASC to handle a variety of new misuse cases and provide many more options for mutations. This allows for more options to evaluate crypto-detectors and determine identify their flaws.
\end{flushleft}
\begin{flushleft}
Finally MASC built a taxonomy of known crypto api misuses up till 2019. This taxonomy was updated to include any misuses discovered between 2019 and 2022. This resulted in 19 new papers being added to the taxonomy with a new total of 55 papers. This expansion both reinforces the misuses that were previously identified in the taxonomy and added several new misuses to the taxonomy as well.
\end{flushleft}
\begin{flushleft}
Finally MASC built a taxonomy of known crypto api misuses up till 2019. This taxonomy was updated to include any misuses discovered between 2019 and 2022. This resulted in 19 new papers being added to the taxonomy with a new total of 55 papers. This expansion both reinforces the misuses that were previously identified in the taxonomy and added several new misuses to the taxonomy as well.
\end{flushleft}
\begin{flushleft}
To analze crypto-dectors Ilooked at both the 9 crypto-detectors evaluated in the original work and 5 new crypto-detectors. For the original crypto-dectors Ievaluated them with the updated MASC against their most up to date versions to determine if old flaws that have previously been fixed have a tendency to reappear. Both old and new crypto-dectors were evaluated with mutated Android and Java applications that were used in the original MASC paper, 15 newly mutated Android and Java applications, and minimal examples of cryptographic misuse.
\end{flushleft}

\end{large}