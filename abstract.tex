\begin{large}
\noindent 
\begin{flushleft}
The correct use of cryptography is central to ensuring data security in modern software systems. Hence, several academic and commercial static analysis tools have been developed for detecting and mitigating crypto-API misuse. While developers are optimistically adopting these crypto-API misuse detectors (or crypto-detectors) in their software development cycles, this momentum must be accompanied by a rigorous understanding of their effectiveness at finding crypto-API misuse in practice. The original paper presents the MASC framework, which enables a systematic and data-driven evaluation of crypto-detectors using mutation testing. MASC was grounded in a comprehensive view of the problem space by developing a data-driven taxonomy of existing crypto-API misuse, containing 105 misuse cases organized among nine semantic clusters. 12 generalizable usagebased mutation operators were developed and three mutation scopes that can expressively instantiate thousands of compilable variants of the misuse cases for thoroughly evaluating crypto-detectors. Using MASC, nine major crypto-detectors were evaluated and 19 unique, undocumented flaws that severely impact the ability of crypto-detectors to discover misuses in practice were found.
\end{flushleft}

\begin{flushleft}
For my thesis I built upon this previous research and greatly expanded the MASC framework. MASC was expanded in all areas by adding new functionality, new operators, new misuses, and expanding the taxonomy. In addition, I reevaluated the most up to date versions of the original 9 crypto-detectors and evaluated 5 additional crypto-dectors. On top of this I also doubled the amount of applications I used to evaluate the tools.
\end{flushleft}
\begin{flushleft}
To analze crypto-dectors I looked at both the 9 crypto-detectors evaluated in the original work and 5 new crypto-detectors. For the original crypto-detectors I evaluated them with the updated MASC against their most up to date versions to determine if old flaws that have previously been fixed have a tendency to reappear. Both old and new crypto-detectors were evaluated with mutated Android and Java applications that were used in the original MASC paper, 15 newly mutated Android and Java applications, and minimal examples of cryptographic misuse.
\end{flushleft}

\end{large}