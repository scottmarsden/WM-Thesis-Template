\makeatletter

% this  is a easy way to add and highlight new text  ...
% just comment in/out the \tnew macro ..

\newcommand{\tnew}[1]{{\bf { #1 }} }
%\newcommand{\tnew}[1]{{ { #1 }} }

% math and theorem definition

\newcommand{\ndef}{\stackrel{\rm def}{=}}

% this is used for draft only

%\renewcommand{\baselinestretch}{2}

% just to number pages in the draft

\useunder{\uline}{\ul}{}

% nothing i.e., no-numbering final and camera ready

%\pagestyle{empty}

\newcommand{\CrashScopesp}{{\textsc CrashScope~}}
\newcommand{\CrashScopes}{{\textsc \small CrashScope's~}}
\newcommand{\CrashDroid}{{\textsc CrashDroid~}}
\newcommand{\CrashScopebf}{{ \textbf{\textsc CrashScope~}}}

\newcommand{\ReDraw}{{\textsc{\small ReDraw}}\xspace}
\newcommand{\ReDraws}{{\textsc{\small ReDraw's}}\xspace}
\newcommand{\Remaui}{{\textsc{\small Remaui}}\xspace}
\newcommand{\Remauis}{{\textsc{\small Remaui's}}\xspace}
\newcommand{\CrashScope}{{\textsc{\small CrashScope}}\xspace}
\newcommand{\MonkeyLab}{{\textsc{\small MonkeyLab}}\xspace}
\newcommand{\pixcode}{{pix2code}\xspace}

\newcommand{\GVTsp}{{\textsc{\small Gvt~}}}
\newcommand{\GVT}{{\textsc{\small Gvt}}}
\newcommand{\GVTs}{{\textsc{\small Gvt's~}}}
\newcommand{\dv}{{\textit{DV}}\xspace}
\newcommand{\dvs}{{\textit{DVs}}\xspace}
\newcommand{\gc}{{\textit{GC}}\xspace}
\newcommand{\gcs}{{\textit{GCs}}\xspace}
\newcommand{\mgc}{{\textit{M-GC}}\xspace}
\newcommand{\mgcs}{{\textit{M-GCs}}\xspace}
\newcommand{\igc}{{\textit{I-GC}}\xspace}
\newcommand{\igcs}{{\textit{I-GCs}}\xspace}


\newcommand{\ReDrawAbs}{{R}{\ssmall E}{D}{\ssmall RAW}\xspace}




\newsavebox\CBox
\newlength\CLength
\def\circled#1{\sbox\CBox{#1}%
  \ifdim\wd\CBox>\ht\CBox \CLength=\wd\CBox\else\CLength=\ht\CBox\fi
    \makebox[1.2\CLength]{\makebox(0,0.9\CLength){\put(0,0){\circle{1.3\CLength}}}%
    \makebox(0,1.0\CLength){\put(-.5\wd\CBox,0){#1}}}}

\def\circledlong#1{\sbox\CBox{#1}%
  \ifdim\wd\CBox>\ht\CBox \CLength=\wd\CBox\else\CLength=\ht\CBox\fi
    \makebox[1.2\CLength]{\makebox(0,0.6\CLength){\put(0,0){\circle{1.3\CLength}}}%
    \makebox(0,0.6\CLength){\put(-.5\wd\CBox,0){#1}}}}

\lstset{
	basicstyle=\footnotesize\ttfamily,
	breaklines=true,
    frame=tb, % draw a frame at the top and bottom of the code block
    tabsize=4, % tab space width
    showstringspaces=false, % don't mark spaces in strings
    numbers=left, % display line numbers on the left
    commentstyle=\color{Red}, % comment color
    keywordstyle=\color{blue}, % keyword color
    stringstyle=\color{OliveGreen}, % string color
	xleftmargin=.25in %align numbers to left side
}

\newboolean{showcomments}


\setboolean{showcomments}{true}

\ifthenelse{\boolean{showcomments}}
  {\newcommand{\nb}[2]{
    \fbox{\bfseries\sffamily\scriptsize#1}
    {\sf\small$\blacktriangleright$\textit{#2}$\blacktriangleleft$}
   }
   \newcommand{\cvsversion}{\emph{\scriptsize$-$Id: macro.tex,v 1.9 2005/12/09 22:38:33 giulio Exp $}}
  }
  {\newcommand{\nb}[2]{}
   \newcommand{\cvsversion}{}
  }


\newcommand{\ie}{\textit{i.e.},\xspace}
\newcommand{\eg}{\textit{e.g.},\xspace}
\newcommand{\etc}{\textit{etc.}\xspace}
\newcommand{\etal}{\textit{et al.}\xspace}
\newcommand{\aka}{\textit{a.k.a.}\xspace}

\newcommand\NEW[1]{\nb{NEW}{#1}}

\newcommand\operator[2]{{\bf OP$_{#1}$: {\em {#2}} -- }}
\newcommand\opnumber[1]{{{\bf OP}$_{#1}$}}

\newcommand{\projtodo}[1]{{\color{red}{\bf TODO:}#1}}

\newcommand*{\img}[1]{%
    \raisebox{-.2\baselineskip}{%
        \includegraphics[
        height=0.85\baselineskip,
        width=0.85\baselineskip,
        keepaspectratio,
        ]{#1}%
    }%
}



\setboolean{showcomments}{true}



\newcommand{\addnewline}{\\}

\newcommand{\cancel}[1]{{\leavevmode\color{RubineRed}{\sout{\xspace#1}}}}
\newcommand{\edit}[2]{{\leavevmode\color{RubineRed}{\sout{#1}}}{\color{blue}{\xspace#2}}}
\newcommand{\editl}[2]{{\leavevmode\color{RubineRed}{\addnewline\sout{#1}}}{\color{blue}{\addnewline\xspace#2\addnewline}}}
\newcommand{\rewrite}[2]{{\leavevmode\color{RubineRed}{\sout{#1}}}{\color{Green}{\arrow\xspace#2}}}


\newcommand{\add}[1]{{\leavevmode\color{blue}{#1}}}
\newcommand{\addamit}[1]{{\leavevmode\color{black}{#1}}}
\newcommand{\addnew}[1]{{\leavevmode\color{black}{#1}}}

\newcommand{\addl}[1]{{\leavevmode\color{Green}{\addnewline+ \xspace#1\addnewline}}}
\newcommand{\remove}[1]{{\leavevmode\color{red}{\xspace#1}}}
\newcommand{\removel}[1]{{\leavevmode\color{red}{\addnewline- \xspace#1\addnewline}}}


\newcommand{\grayme}[1]{{\leavevmode\color{gray}{\xspace#1}}}


\newcommand{\recycler}{\textsf{RecyclerView}\xspace}


\newcommand\finding[1]{\vspace{0.25em}\noindent\textsf{\bf Finding {#1}.}}
\newcommand\fnumber[1]{{\bf F{#1}}}
\newcommand\opnumbernormal[1]{{{OP}$_{#1}$}}



\newcommand{\arrow}{{$\rightarrow$}\xspace}
\newcommand\inline[1]{{\lstinline[keywordstyle=\color{black},basicstyle=\scriptsize\ttfamily,stringstyle=\color{black}]{#1}}}
\newcommand\inlinesmall[1]{{\lstinline[keywordstyle=\color{black},basicstyle=\small\ttfamily,stringstyle=\color{black}]{#1}}}



\newcommand{\boxme}[1]{{
\begin{tcolorbox}[enhanced,skin=enhancedmiddle,borderline={1mm}{0mm}{MidnightBlue}]
    \textbf{Insight: } #1 \end{tcolorbox}
}}



\newcommand\fix[1]{{\color{blue} \nb{FIX THIS}{#1}}}
\newcommand\blue[1]{{\color{blue}{#1}}}
\newcommand{\here}{{\color{blue} \nb{***}{CONTINUE HERE}}}
\newcommand{\REF}{{\color{red} \textbf{[REFS]}}\xspace}
\newcommand{\xy}{{\color{red} \textbf{XY}}\xspace}
\newcommand\tops[1]{{\color{blue}{#1}}}
\newcommand\alert[1]{{\color{red}{#1}}}


\newcommand{\target}{\textit{target tool}\xspace}
\newcommand{\targets}{\textit{target tools}\xspace}
\newcommand{\behavior}{\textit{target behavior}\xspace}


\newcommand{\mplus}{{\sc MDroid+}\xspace}

\newcommand{\secref}[1]{\S\ref{#1}\xspace}
\newcommand{\figref}[1]{Fig.~\ref{#1}\xspace}
\newcommand{\tabref}[1]{Table~\ref{#1}\xspace}
\newcommand{\Phase}{{\sc Phase}\xspace}
\newcommand{\Phases}{{\sc Phase's~}\xspace}

\newcommand{\emphquote}[1]{{\emph{`#1'}}\xspace}
\newcommand{\emphdblquote}[1]{{\emph{``#1''}}\xspace}

\newcommand{\emphbrack}[1]{\emph{[#1]}\xspace}

\newcommand{\subj}{\emphbrack{subject}}
\newcommand{\act}{\emphbrack{action}}
\newcommand{\obj}{\emphbrack{object}}
\newcommand{\prep}{\emphbrack{preposition}}
\newcommand{\objtwo}{\emphbrack{object2}}

\newcommand*\ciclednum[1]{\raisebox{.5pt}{\textcircled{\raisebox{-.9pt}
{#1}}}}

\newcommand\codel[1]{\begin{verbatim}{#1}\end{verbatim}}









